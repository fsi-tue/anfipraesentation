% !TeX document-id = {fc3ccb1e-aeba-4c77-9990-74eef049f67c}
% !TeX TXS-program:compile = txs:///pdflatex/[-output-directory=tmp]

%============================ TODO: UPDATE =====================================
% - Update the following dates, links
% - (Update 'pictures/erstiheft.pdf')
%- - - - - - - - - - - - - - - - - - - - - - - - - - - - - - - - - - - - - - - -
\newcommand{\semester}{         WiSe 24}
\newcommand{\dateCHF}{          (31.10.)}
%         ⚠ COMMENT OUT EVERY NOT-OCCURING EVENT STARTING LINE 250 ⚠
% \newcommand{\dateSommerfest}{   (28.06.)}
\newcommand{\dateSpieleabend}{  (01.10., 10.10., 17.10.)}
\newcommand{\dateGrillen}{      (12.10., 19.10.)}
%\newcommand{\dateStadtrallye}{  (20.10.)}
\newcommand{\dateKneipentour}{  (08.10., 14.10.)}
\newcommand{\dateWanderung}{    (05.10., 26.10.)}
\newcommand{\dateFilmabend}{    (15.10.)} 
\newcommand{\dateFlunkyball}{   (02.10.)}
\newcommand{\dateCTF}{          (20.10.)}
\newcommand{\dateKastenlauf}{   (16.10.)}
%- - - - - - - - - - - - - - - - - - - - - - - - - - - - - - - - - - - - - - - -
\newcommand{\timeSitzung}{Donnerstags 18:30 Uhr}
\newcommand{\locFachschaftszimmer}{C125 - Sand 14}
\newcommand{\locFSKammer}{C102 - Sand 14}
% \newcommand{\linkBBB}{https://bbb.fsi.uni-tuebingen.de/b/luk-v3t-dvk}
%- - - - - - - - - - - - - - - - - - - - - - - - - - - - - - - - - - - - - - - -
\newcommand{\linkMaillists}{https://www.fsi.uni-tuebingen.de/infos/maillists/}
\newcommand{\linkErsti}{https://www.fsi.uni-tuebingen.de/du-bist-ersti/}
\newcommand{\linkAnmeldung}{https://eei.fsi.uni-tuebingen.de/}
\newcommand{\linkErstiheft}{https://teri.fsi.uni-tuebingen.de/anfiheft/}
\newcommand{\linkDiscord}{https://discord.gg/d4X2WjHWmQ}
\newcommand{\linkPresentation}{https://www.fsi.uni-tuebingen.de/anfip}
\newcommand{\linkWebsiteInfo}{https://www.fsi.uni-tuebingen.de}
\newcommand{\linkWebsiteKogni}{https://www.fs-kogni.fsi.uni-tuebingen.de}
%===============================================================================

\documentclass[aspectratio=169]{beamer}

%-------------------------------------------------------------------------------
%   Edit colour scheme 
%-------------------------------------------------------------------------------



%===============================================================================

\beamertemplatenavigationsymbolsempty
\usepackage[utf8]{inputenc}
\usepackage{DejaVuSans}
\renewcommand*\familydefault{\sfdefault}
\usepackage[fixed]{fontawesome5}

\useoutertheme{infolines}
\useinnertheme[realshadow,corners=2pt,padding=2pt]{chamfered}
\usecolortheme{fsik}
\usepackage[ngerman]{babel}
\usepackage{graphics, graphicx}
\usepackage{color}
\usepackage{latexsym}
\usepackage{etoolbox}
\usepackage{subfig}
\usepackage{tikz}
\PassOptionsToPackage{hyphens}{url}
\usepackage{hyperref}
\usepackage{url}
\usepackage{qrcode}
\usepackage{eurosym}

% Override these to reproduce a build, if you run
% "nix-build --argstr date YYYY-MM-DD" you don't need to change them manually:
\year=\year
\month=\month
\day=\day

% For a presenter mode viewer like 'pdfpc' invoked with 'make notes'
\newif\ifnotes
\input{makeconfig.tex}
\ifnotes
    \usepackage{pgfpages}
    \setbeameroption{show notes on second screen}
\fi


\newcommand<>{\hover}[1]{\uncover#2{%
		\begin{tikzpicture}[remember picture,overlay]%
		\draw[fill,opacity=0.4] (current page.south west)
		rectangle (current page.north east);
		\node at (current page.center) {#1};
		\end{tikzpicture}}
}

\setbeamertemplate{headline}{}

\definecolor{pblue}{rgb}{0.13,0.13,1}
\definecolor{pgreen}{rgb}{0,0.5,0}
\definecolor{pred}{rgb}{0.9,0,0}
\definecolor{pgrey}{rgb}{0.46,0.45,0.48}
\definecolor{darkgreen}{rgb}{0,0.5,0}
\definecolor{fsiblue}{HTML}{000080}
\definecolor{fsikturqoise}{HTML}{008080}
\definecolor{fskgreen}{HTML}{008000}

\title{Fachschaftsvorstellung}
\author{\textbf{fsik}}

\begin{document}
    
%------------------------------------------------------------------------------
%   Title
%------------------------------------------------------------------------------
\begin{frame}
    \centering
    \vspace{.5cm}
	\includegraphics[width=0.6\textwidth]{pictures/fsilogo_neu.pdf}\\
	\includegraphics[width=0.6\textwidth]{pictures/fsklogo.pdf}\\
	\vspace*{1cm}
	\begin{Large}
	   Hey! Wir helfen euch $\{\text{ins}\mid\text{im}\}$ Studium!
	\end{Large}

    \vspace{1cm}
    \semester
\end{frame}

%= = = = = = = = = = = = = = = = = = = = = = = = = = = = = = = = = = = = = = = =
%   Table of Contents
%   =================
%
%   1. Wer sind wir?
%   2. Was machen wir?
%   3. Mailinglisten
%   4. Wie lernt man Leute kennen
%       4.1 Ersti-Programm
%       4.2 Erstiheft
%       4.3 Discord
%   5. Mach doch mit!	
%= = = = = = = = = = = = = = = = = = = = = = = = = = = = = = = = = = = = = = = =


%-------------------------------------------------------------------------------
%   Wer sind wir?
%-------------------------------------------------------------------------------

%\section{Wir, die FSI!}
%\begin{frame}{\insertsection}
%    \begin{figure}
%        \includegraphics[width=\linewidth]{pictures/fsi_22.jpg}
%    \end{figure}
%
%    \note{Hier labern wir sympathisch alá ''Das da bin ich ohne Maske'', 
%          ''Das ist der Tim'', ''Oh guck! Grace!'' usw.}
%\end{frame}



%-------------------------------------------------------------------------------
%   Was machen wir? Wir vertreten euch!
%-------------------------------------------------------------------------------

\section{Was machen wir?}
\begin{frame}{\insertsection}
    \begin{columns}
        \column{.5\textwidth}
        {\huge\itshape \textcolor{fsikturqoise}{Wir vertreten euch!}}
        
        \column{.5\textwidth}
        \begin{itemize}
            \item \textbf{Beratung/Betreuung} bei deinen Problemen
            \note[item]{Bei Fragen und Problemen an die Fachschaft wenden}

            \item \textbf{Kontakt} zur Professorenschaft
            \note[item]{FBI erwähnen aka monatl. Treffen 
                        \& FB Leitung ist uns aufgeschlossen}

            \item \textbf{Aktionen} und \textbf{Events}
                  \faMusic Clubhausfest \dateCHF\ und \\
                  \faBeer Weihnachtsfeier (TBA)
                  %\faBeer Sommerfest \dateSommerfest
            \note[item]{Neben Ansprechsperson sind wir für Alkoholausschank da}

%            \item \textbf{Hochschulpolitik}: StuRa, FakRat
%            \item \textbf{Gremien} wie Studienkommission, Prüfungsausschuss,
%                  Berufungskommissionnen
%            \note[item]{Im PA sitzen Leute, die bei Härtefallanträgen auf deiner
%                        Seite sind. Evaluationen werden in Stuko besprochen,...}
%
%            \item \textbf{Tools \& Services} wie PPI, KKI, BBB, CodiMD
%            \note[item]{Alktklausuren auf \texttt{ppi.fsi.uni-...}, Kneipen auf 
%                        \texttt{kki.fsi.uni-...}, Eigene VideoConf auf 
%                        \texttt{bbb.fsi.uni-...}, Gemeinsame Notizen auf 
%                        \texttt{notes.fsi.uni-...}}
        \end{itemize}
        
    \end{columns}
\end{frame}


%-------------------------------------------------------------------------------
%   Mailinglisten
%-------------------------------------------------------------------------------

%\section{Mailinglisten}
%\begin{frame}[t]{\insertsection}
%    \textbf{Erstens:} Lest eure Uni-Mails!
%    \footnote{also die wichtigen. Wie ihr Spam" wegfiltert haben wir für euch hier:
%              \url{https://github.com/fsi-tue/workshops/blob/master/2019/05-8-unimails.pdf}}
%    \begin{itemize}
%        \item \textbf{info-studium}\\
%              \textcolor{red}{\faExclamationTriangle 
%                  Der einzige Weg des Fachbereich euch wichtige Informationen 
%                  zukommen zu lassen. Anmeldung ist \textbf{PFLICHT}!
%              \faExclamationTriangle}
%        \item \textbf{info-talk}\\
%              Für Informationen rund um Tübingen, der Informatik und dem Rest.
%        \item \textbf{info-jobs}\\
%              Jobangebote mit Bezahlung
%    \end{itemize}
%    \vfill
%    \textbf{Anmeldung:} \url{\linkMaillists}
%
%    \note{Das hier ist \textit{arguably} die wichtigste Folie in der Präsentation.\\
%          Die einzige Möglichkeit von der Anmeldung zum \textbf{Praktikum Mikrocomputer}
%          und dem \textbf{Teamprojekt} zu erfahren. Letztere ist nur einmal im 
%          Jahr.\\
%          Oder wenn der Prüfungsausschuss Informationen verschicken will.\\[2em]
%          info-talk für Studenten-Kosmos und info-jobs Jobangebote, beide
%          moderiert.}
%\end{frame}
%
%\begin{frame}{Direkt anmelden!}
%    \begin{columns}
%        \column{.7\textwidth}
%        \begin{enumerate}
%            \item Auf die Seite gehen
%            \item Im Textfeld oben eure \textbf{Email-Adresse} eintragen
%            \item Den Button \textbf{\textsc{REGISTRIEREN}} anklicken
%        \end{enumerate}
%    
%        \centering
%        \vspace{1cm}
%        {\huge\color{fsiblue} \faHandsWash Danke allerseits! }
%                
%        \column{.3\textwidth}
%        \centering
%        \qrcode[height=.9\linewidth]{\linkMaillists}
%        \\[1em]
%        \url{\linkMaillists}
%    \end{columns}
%    \note{Hier jetzt Zeit geben, dass sich die Erstis anmelden können}
%\end{frame}


%-------------------------------------------------------------------------------
%   Wie lerne ich leute kennen?
%-------------------------------------------------------------------------------

\begin{frame}{Erstiheft}
    \begin{columns}
        \column{.7\textwidth}
        \begin{figure}
            \colorbox[HTML]{c3e6f1}{
                \includegraphics[height=.8\textheight]{pictures/erstiheft.pdf}}
        \end{figure}
        
        \column{.3\textwidth}
        \centering
        \qrcode[height=.9\linewidth]{\linkErstiheft}
        \\[1em]
        \url{\linkErstiheft}
    \end{columns}

    \note{Hier kann auch sehr gut darauf hingewiesen werden, dass alle unsere
          Materialien online abrufbar sind.}
\end{frame}

%- - - - - - - - - - - - - - - - - - - - - - - - - - - - - - - - - - - - - - - -
% TODO: Might replace with generic version without dates
%- - - - - - - - - - - - - - - - - - - - - - - - - - - - - - - - - - - - - - - -
\subsection{Ersti-Programm}

\begin{frame}{\insertsubsection}
    \begin{columns}
        \column{.7\textwidth}
        \begin{itemize}
            \large
            \item {\large\faDice} Spieleabend  \dateSpieleabend
			\item {\large\faBeer} Flunkyball   \dateFlunkyball
            \item {\large\faHiking} Wanderung  \dateWanderung
            \item {\large\faBeer} Kneipentour  \dateKneipentour
            \item {\large\faUtensils} Grillen  \dateGrillen
            %\item {\large\faRoute} Stadtrallye \dateStadtrallye
			\item {\large\faFilm} Filmabend    \dateFilmabend
			\item {\large\faBeer} Kastenlauf   \dateKastenlauf
			\item {\large\faFlag} Capture the Flag \dateCTF
        \end{itemize}
        \centering
        \vspace{1em}
        {\huge\color{fsikturqoise} Anmeldung \faLongArrowAltRight}
        
        \column{.3\textwidth}
        \centering
        \qrcode[height=.9\linewidth]{\linkAnmeldung}
        \\[1em]
        \url{\linkAnmeldung}
    \end{columns}

    \note{Hier lässt sich darauf hinweisen, dass die Anmeldung zwecks Planbarkeit
          sehr gewünscht ist.}
\end{frame}



%- - - - - - - - - - - - - - - - - - - - - - - - - - - - - - - - - - - - - - - -
%\subsection{Sonstiges}
%\begin{frame}{Discord anyone?}
%    \begin{columns}
%        \column{.7\textwidth}
%        \centering
%        \textit{\Large Mit Leuten quatschen, Online-Spieleabende und 
%                       Übungsblätter remote machen?}\\[1em]
%        {\Huge\color{fsikturqoise}\faComments}
%        
%%        
%        \column{.3\textwidth}
%        \centering
%        \qrcode[height=.9\linewidth]{\linkDiscord}
%        \\[1em]
%        \url{\linkDiscord}
%    \end{columns}
%
%    \note{Kommunikation ist wichtig; Wurde von nem Studi eröffnet, wir haben 
%          uns da rangehangen; usw u so fort...}
%\end{frame}


%-------------------------------------------------------------------------------
%   Mach doch mit!
%-------------------------------------------------------------------------------

\section{Mach doch mit!}
\begin{frame}{\insertsection}
    \Large
    {\Huge\color{fsikturqoise} \faClock[regular]}   \timeSitzung \\[.5em]
    {\Huge\color{fsiblue} \faMapMarker*}       \locFachschaftszimmer\\[.5em]
    {\Huge\color{fskgreen} \faMapMarker*}       \locFSKammer\\[.5em]
    \color{darkgray}    % TODO: Might be removable for future screenings
    % {\Huge\color{fsiblue!50!darkgray} \faMapMarker*} \url{\linkBBB}
\end{frame}

%-------------------------------------------------------------------------------
%   Link zu dieser Präsentation
%-------------------------------------------------------------------------------

%\section{Diese Präsentation zum selber durchklicken}
%\begin{frame}{\insertsection}
%    \begin{columns}
%        \column{.5\textwidth}
%        \centering
%        \includegraphics[height=.8\textheight]{pictures/keepcalm.pdf}
%
%        \column{.5\textwidth}
%        \centering
%        \qrcode[height=.5\textheight]{\linkPresentation}
%        \\[1em]
%        \url{\linkPresentation}
%
%    \end{columns}
%\end{frame}

\section{Mehr Informationen...}
\begin{frame}{\insertsection}
    \begin{columns}
		\column{.3\textwidth}
        \centering
        \textcolor{fsiblue}{\qrcode[height=.5\textheight]{\linkWebsiteInfo}
        \\[1em]
        \url{\linkWebsiteInfo}} 
    
        \column{.4\textwidth}
        \centering
        \includegraphics[height=.8\textheight]{pictures/keepcalm_fsik2.pdf}
        
        \column{.3\textwidth}
        \centering
        \textcolor{fskgreen}{\qrcode[height=.5\textheight]{\linkWebsiteKogni}
        \\[1em]
        \url{\linkWebsiteKogni}}

    \end{columns}
\end{frame}

\end{document}
