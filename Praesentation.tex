\documentclass{beamer}
\beamertemplatenavigationsymbolsempty

\usepackage[utf8]{inputenc}
\usepackage[ngerman]{babel}
\usepackage{graphicx}
\usepackage{pdfpages}
\usepackage{hyperref}

\title{Fachschaft Informatik}

\begin{document}
	\maketitle

%	\begin{frame}{Warum fsi?}
%		\begin{itemize}
%			\item Der Verlauf eures Studiums ist bestimmt durch:
%				\begin{itemize}
%					\item Modulhandbuch
%					\item Prüfungsordnung
%				\end{itemize}
%			\item Wenn etwas schief läuft müsst ihr zum
%				\begin{itemize}
%					\item Prüfungsamt
%					\item Prüfungsausschuss
%				\end{itemize}
%		\end{itemize}
%	\end{frame}

	\begin{frame}{Unsere offiziellen Aufgaben}
		\framesubtitle{Gremienarbeit}
		Wir vertreten euch
		\begin{itemize}
			\item im Fakultätsrat (FakRat)
			\item in den Studienkomissionen (StudKomm)
			\item in den Prüfungsausschüssen (PA)
			\item in der Fachschaftenrätevollversammlung (FSRVV)
			\item ... und in vielen anderen Gremien und Kommissionen
			\item[$\rightarrow$] www.fsi.uni-tuebingen.de/fachschaft/vertreter
		\end{itemize}
		Durch einen ``guten Draht'' zu den Professoren können wir Probleme aufzeigen und gegebenenfalls vermitteln.
	\end{frame}

	% %Nicht in der VL

	% \begin{frame}{Fakultätsrat}
	% 	Entscheidet über
	% 	\begin{itemize}
	% 		\item Finanzen
	% 		\item neue Professuren
	% 		\item Forschung und Lehre
	% 		\item Besetzung von Gremien
	% 	\end{itemize}
	% 	Wird jährlich von allen Studierenden der Fakultät gewählt (5 Studenten aus 11 Fachschaften).
	% \end{frame}

	% \begin{frame}{Studienkommission}
	% 	\begin{itemize}
	% 		\item Wichtigstes Gremium für den Studienablauf
	% 		\item Gestaltet Modulhandbuch
	% 		\item Zwei Kommissionen:
	% 			\begin{itemize}
	% 				\item (Bio-, Medien-, Medizin-) Informatik
	% 				\item Kognitionswissenschaft
	% 			\end{itemize}
	% 		\item 4 Profs, 2 Mitarbeiter, 4 Fachschaftler\\
	% 			$\Rightarrow$ großer Einfluss der fsi
	% 	\end{itemize}
	% \end{frame}

	% \begin{frame}{Prüfungsausschuss}
	% 	\begin{itemize}
	% 		\item Anerkennung von Studienleistungen
	% 			\begin{itemize}
	% 				\item Auslandssemester?
	% 				\item ungewöhnliche Vorlesung für ein bestimmtes Modul?
	% 				\item neues Nebenfach?
	% 			\end{itemize}
	% 		\item (theoretisch) PA für jeden Studiengang einzeln
	% 		\item Fachschaftler mit beratender Stimme
	% 	\end{itemize}
	% \end{frame}

	% \begin{frame}{Hochschulpolitik}
	% 	\begin{itemize}
	% 		\item StuRa - Studierendenrat (ersetzt ASTA)
	% 			\begin{itemize}
	% 				\item Hochschulpolitisches Mandat
	% 				\item kann Projekte unterstützen
	% 				\item Gebühren zu Beginn des Semesters
	% 			\end{itemize}
	% 		\item Senat
	% 		\item Vertreter fühlen sich meist einer Hochschulpolitischen Gruppe zugehörig:
	% 			\begin{itemize}
	% 				\item FSVV
	% 				\item GHG
	% 				\item JuSos
	% 				\item RCDS
	% 				\item ...
	% 			\end{itemize}
	% 	\end{itemize}
	% \end{frame}

	% \begin{frame}{Was wir sonst noch für euch tun}
	% 	\begin{itemize}
	% 		\item Webseite – Zentraler Informationsknotenpunkt
	% 		\item Mailinglisten – Austausch unter den Studenten
	% 		\item Prüfungsprotokolle –\\
	% 			Was fragt ein Prüfer eigentlich so in der Prüfung?\\
	% 			Geschrieben von Studenten für Studenten.
	% 		\item Anfängerbetreuung: Tipps fürs Studium, Frühstück, Kneipentour, Anfängerwochenende,...
	% 		\item Partys und Feste organisieren :-)
	% 	\end{itemize}
	% \end{frame}

	%Nur VL
	\begin{frame}{Was wir sonst noch für euch tun...}
	\framesubtitle{Sommerfest}
		\begin{columns}
			\begin{column}{.4\linewidth}
				\begin{itemize}
					\item einmal im Jahr
					\item immer Ende Juni / Anfang Juli
				\end{itemize}
			\end{column}
			\begin{column}{.7\linewidth}
				Letztes Semester:\vspace*{2mm}\\
				\includegraphics[width=\linewidth]{sf15.jpg}
			\end{column}
		\end{columns}
	\end{frame}
	
		
	\begin{frame}{ClubHausFest}
		\begin{columns}
			\begin{column}{.5\linewidth}
				\begin{itemize}
					\item immer donnerstags
					\item im Clubhaus
					\item pro Fachschaft 1x im Semester
					\item zusammen mit der Fachschaft Psychologie
					\item dieses Semester 19.11.2015
				\end{itemize}
			\end{column}
			\begin{column}{.5\linewidth}
				\includegraphics[width=\linewidth]{CHF_Flyer.png}
			\end{column}
		\end{columns}
	\end{frame}
	
	\begin{frame}{Ersti-Veranstaltungen}
		\begin{itemize}
			\item Grillfest zum Kennenlernen (07.10.)
			\item Anfifrühstück mit Führung über Morgenstelle und Sand (09.10.)
			\item Kneipentour (09.10.)
			\item Stadtführung (12.10.)
			\item Spieleabend auf dem Sand (14.10., 19 Uhr, Raum A104)
			\item Semesteropeningparty (16.10., 20 Uhr, Raum A104)
			\item Wanderung nach Bebenhausen \\ (18.10., 11 Uhr, Sand Haupteingang, Mittagessen bitte mitbringen) 
			\item Anfängerwochenende
		\end{itemize}
	\end{frame}


	% % Nicht in VL
	% \begin{frame}
	% 	\begin{center}
	% 		\includegraphics[scale=0.45]{CHF_Flyer.png}
	% 	\end{center}
	% \end{frame}


	% \begin{frame}{Sommerfest}
	% 	\begin{columns}
	% 		\begin{column}{.5\linewidth}
	% 			\begin{itemize}
	% 				\item auf dem Sand
	% 				\item Volleyballspiel fsi vs. Profs
	% 				\item jedes Sommersemester (Ende Juni / Anfang Juli)
	% 			\end{itemize}
	% 		\end{column}
	% 		\begin{column}{.5\linewidth}
	% 			\includegraphics[width=\linewidth]{sf10.png}
	% 		\end{column}
	% 	\end{columns}
	% \end{frame}

	% {\setbeamercolor{background canvas}{bg=black}
	% \begin{frame}
	% 	\begin{center}
	% 		\includegraphics[height=.95\paperheight]{sf13.png}
	% 	\end{center}
	% \end{frame}
	% }
	
	\begin{frame}{Anfi-Wochenende}
		\framesubtitle{auf Schachen bei Münsingen}
		\begin{itemize}
			\item Freitag, den 06. November bis Sonntag, den 08. November 2015
			\item Anmeldung seit 13.10., 18 Uhr per E-Mail
			\item Teilnehmerzahl ist begrenzt (40 Plätze)
			\item weitere Informationen auf \url{http://www.fsi.uni-tuebingen.de/erstsemester/veranstaltungen/wochenende}
		\end{itemize}
	\end{frame}

	\begin{frame}[<+->]{Mailinglisten}
		\begin{itemize}
			\item info-studium
			\item kogwiss
			\item versuche
			\item info-jobs
			\item info-talk
			\item info-alumni
            \item sport
			\item Anmeldung:\\
				Mail an \$LISTNAME-subscribe@fsi.uni-tuebingen.de
		\end{itemize}
	\end{frame}

	\begin{frame}{Fachschaftsarbeit - wie geht das?}
		\begin{columns}
			\begin{column}{.5\linewidth}
				\begin{itemize}
					\item jeden Donnerstag 18:30 Uhr Sitzung
					\item Raum A104
					\item nächste Sitzung: \\Donnerstag 22.10. 18:30 Uhr
					\item Kommt vorbei!
				\end{itemize}
			\end{column}
			\begin{column}{0.5\linewidth}
				\includegraphics[width=\linewidth]{Lageplan.png}\\
				\includegraphics[width=\linewidth]{Sitzung.jpg}
			\end{column}
		\end{columns}
	\end{frame}
	
	\begin{frame}{Studienberatung}
			\begin{itemize}
				\item Informatik/Bioinformatik
					\begin{itemize}
					\item Jessica Wolz
					\item studienberatung@informatik.uni-tuebingen.de
					\end{itemize}
				\item Medieninformatik
				\begin{itemize}
					\item Fiete Botschen
					\item medieninformatik@uni-tuebingen.de
				\end{itemize}
				\item Kognitionswissenschaften
					\begin{itemize}
						\item Felicia Saar
						\item kogni-beratung@fsi.uni-tuebingen.de
					\end{itemize}
			\end{itemize}
	\end{frame}


	\begin{frame}
		%\includegraphics[width=\linewidth]{fsi-manu.png}
		\includegraphics[width=\linewidth]{volleyball_sofe15.jpg}
		\vspace*{2mm}\\
		{\centering\url{www.fsi.uni-tuebingen.de}}
	\end{frame}


\end{document}
